\documentclass[11pt]{article}

\usepackage[utf8]{inputenc}
\usepackage[T1]{fontenc}
\usepackage{graphicx}
\usepackage{longtable}
\usepackage{float}
\usepackage{wrapfig}
\usepackage{soul}
\usepackage{amssymb}
\usepackage{hyperref}
\usepackage{mysymbols}
\usepackage{amsmath}
\usepackage{biblatex}
\bibliography{defeo}

\title{Notes on the theory behind all this}

\begin{document}
\maketitle

\setcounter{tocdepth}{3}
\tableofcontents
\vspace*{1cm}

\section{Towers of extensions from algebraic groups}

Let $\F_q$ be a finite field, let $\ell$ be a prime not dividing $q$,
our goal is to construct the $\ell$-adic closure of $\F_q$ by giving
efficient (quasi-linear) algorithms to perform all the basic
operations (arithmetic, going up and down, but also traces,
frobeniuses, etc.) in it, much in the same spirit as our previous work
on Artin-Schreier towers \cite{df+schost12}.

We start from two works. One is De Smit's and Lenstra's canonical
construction of $\bar{\F}_q$, based on the theory of cyclotomic
extensions and Gauss periods \cite{lenstra+desmit08-stdmodels}. The
other one is Couveignes' ans Lercier's algorithm to generate
irreducible polynomials from isogeny fibers of elliptic curves
\cite{couveignes+lercier11}. While on the surface these two works may
seem very different in spirit, they are deeply linked by a brillant
intuition of Couveignes and Lercier.

Roughly speaking, the idea goes as follows. Take an algebraic group
$G/F_q$ of cardinality $n$ such that $\ell$ divides $n$ (for example
an elliptic curve). Construct an $\F_q$-rational ``isogeny''
$\phi:G\to G'$ of degree $\ell$ with non-trivial kernel in
$\F_q$. Because the kernel is non trivial over $\F_q$, and because $G$
and $G'$ have the same cardinality, the map $\phi$ restricted to
$\F_q$ must necessarily miss some point of $G'$. Let $A$ be any such
point of $G'$, then its fiber contains $n$ points defined over a
finite extension of $\F_q$. If we are ``lucky'', the action of the
Galois group of $\F_q$ over the fiber is transitive, and thus the
fiber must be an irreducible zero-dimensional set. Now one just needs
to compute the polynomials defining the map $\phi$, write down the
system of equations $\phi(x,y,\ldots)=A$ and compute a univariate
representation of it. The resulting polynomial will necessarily be
irreducible.

Couveignes and Lercier show how to use the same idea in two different
cases. When $\ell$ divides $q-1$, they consider the group
$\mathbb{G}_m/F_q$ together with the map $\phi:X\mapsto X^\ell$. When
$\ell$ does not divide $q-1$, they take an elliptic curve with
cardinality divisible by $\ell$. Because of the constraints, the
discriminant of the endomorphism algebra is not divisible by $\ell$,
thus the isogeny volcano of degree $\ell$ is reduced to a cycle. There
are two directions to travel this cycle: one taking isogenies with
rational kernels (corresponding to the Frobenius eigenvalue $1$), and
one taking their duals (corresponding to the Frobenius eigenvalue
$q$). Using complex multiplication, they show that the isogenies in
the first direction have irreducible fibers (the proof does not work
for their duals, and indeed I have verified that the statement does
not hold in general). Furthermore, these isogenies are easy to compute
using Vélu's formulas, and their fibers are easily described as the
roots of a univariate polynomial. This construction can be iterated
over the whole cycle (eventually running through it more than once),
thus obtaining a tower of extensions of degree $\ell$.

\textbf{Note:} I am not sure whether there is a reasonable definition
for \emph{isogeny} outside the world of abelian varieties. Desirable
properties are that it is an algebraic map, a group morphism, it has
finite kernel and is surjective over the algebraic closure. The
$\ell$-th power in $\mathbb{G}_m$ definitely has all the required
properties.

\textbf{Note:} It is hard to imagine a general statement for these
types of constructions based on algebraic groups. The fact that they
may fail in the elliptic case, depending on which kind of isogeny is
taken, leaves little hope of finding such statement. Couveignes and
Lercier also apply a similar construction for $\ell=p$, but no
algebraic group seems to be involved here.

   
\subsection{Using algebraic tori}

The problem with Couveignes' and Lercier's construction is that
finding an elliptic curve with the required cardinality is expensive:
a good fraction of all elliptic curves will have cardinality divisible
by $\ell$ only if $\ell\ll q^{1/4}$, and for each trial curve one
needs to perform point counting. Thus it is natural to ask whether
there are other algebraic groups that can be used in the construction.

Algebraic tori are a natural choice, they indeed are subgroups of
$\mathbb{G}_m$. Useful references are
\cite{rubin+silverberg03,rubin-silverberg+crypto03}

\textbf{Definition:} An algebraic torus $T$ over $\F_q$ is an
algebraic group defined over $\F_q$ that over some finite extension
field is isomorphic to $(\mathbb{G}_m)^d$ , where $\mathbb{G}_m$ is
the multiplicative group and $d$ is necessarily the dimension of
$T$. If $T$ is isomorphic to $(\mathbb{G}_m)^d$ over $\F_{q^n}$ , then
one says that $\F_{q^n}$ splits $T$ .

Define $T_n$ as the subgroup 

\begin{equation}
  T_n = \{\alpha \in F_{q^n} | N_{\F_q/k}(\alpha) = 1 \text{ for any } \F_q\subset k\subset\F_{q^n}\}
\end{equation}

\begin{itemize}
\item $T_n$ is a torus (more precisely, its Weil descent to $\F_q$),
\item the dimension of $T_n$ is $\phi(n)$,
\item the cardinality of $T_n$ is $\Phi_n(q)$, where $\Phi_n$ is the
  $n$-th cyclotomic polynomial,
\item if $h\in T_n(\F_q)$ is an element of prime order not dividing
  $n$, then $h$ does not lie in a proper subfield of $\F_{q^n}/\F_q$.
\end{itemize}

Obviously, $T_n$ is an algebraic variety embedded in $\mathbb{A}^n$,
but it is natural to ask whether it could be embedded in a smaller
space. A torus $T$ of dimension $d$ is said to be \emph{rational} if
there is a birational map $\rho : T\to \mathbb{A}_d$ defined over
$\F_q$.

\textbf{Conjecture (Voskresenskii \cite{voskresenskii98}):} the torus
$T_n$ is rational.

This conjecture is known to be true for $n$ a prime power or the
product of two prime powers. Note, however, that a birational mapping
may \emph{miss} points, and thus it may not be a group morphism. A
computational gain from the rationality of $T_n$ can be hoped for only
if the mapping is a group morphism.

Why are we interested in algebraic tori? Consider the case of $\ell$
not dividing $q-1$. Then it is not possible to use Couveignes' and
Lercier's construction over $\mathbb{G}_m/\F_q$, indeed the
multiplication by $\ell$ map has trivial kernel in this case. However,
if $n|\ell-1$ is the order of $q$ modulo $\ell$, we can apply
Couveignes' and Lercier's algorithm to $\mathbb{G}_m/\F_{q^n}$. We can
use a smaller group by taking $T_n\subset\F_{q^n}^\ast$, indeed $\ell$
divides $\card T_n=\Phi_n(q)$ and this is in general the smallest
subgroup with such property.
    
Parameterizing $T_1$ is trivial, as this is just $\F_q^\ast$. The
first interesting case is $T_2$, which has dimension $1$ and
cardinality $q+1$. We give two equivalent parameterizaions.
    
The first one is given by the Pell conic. Let $d\in\F_q$ be a
quadratic non-residue, the Pell conic is the affine curve
    
\begin{equation}
  C\;:\;X^2 - dY^2 = 1.
\end{equation}
    
If we fix any point on the conic, we have a group law given by the
following construction. To sum two points $A$ and $B$, take the line
through them. Take the parallel line to this passing through the
distinguished point, the second point of intersection is $A+B$. On the
complex unit circle, this law is equivalent to the classical one. If
we fix the point \$(1,0), the group law is given by the following
equation.
    
\begin{equation}
  (r,s) \oplus (t,u) = (rt+dsu, ru+st)
\end{equation}
    
Let $\ell$ divide $q+1$. From the group law, it is not hard to write
the polynomials defining the multiplication-by-$\ell$ map. Let
    
\begin{equation}
  [n](x,y) = (Ax + B, 2Ay)
\end{equation}
   
for some $A,B\in\F_q[x]$, then
   
\begin{align}
  &[n+1](x,y) = (2Ax^2 + Bx -A, (2Ax + B)y),\\
  &[2n](x,y) = (2A^2x^2 + 2ABx + B^2 - A^2, (2A^2x + 2AB)y).
\end{align}
   
Using these two equations, it is immediate to compute $[\ell]$ using
$O(\Mult(\ell))$ operations by double-and-add. Call $f_\ell$ the
polynomial defining the action of $[\ell]$ on the abscissa, then
$f_\ell - a$ is irreducible if and only if $a$ is the abscissa of a
point of maximal $\ell^\alpha$-order. \textbf{Note:} of course, we
need a proof here, but this doesn't seem too hard.

The following Sage code implements the algorithm devised above (and
even a bit more, dealing with $\ell$ non necessarily prime dividing
$q\pm 1$).


\begin{verbatim}
def genpell(K, d):
    primes = d.prime_divisors()
    q = K.cardinality()

    if all(p.divides(q-1) for p in primes):
        qres = True
        C = q - 1
    elif all(p.divides(q+1) for p in primes):
        qres = False
        C = q + 1
    else:
        raise RuntimeError('The prime divisors of d must divide q-1 or q+1.')

    P = PolynomialRing(K, 'x')

    # The 0-division polynomial is 1
    a, b = P(0), P(1)
    for c in reversed(d.digits(2)):
        # double
        a2, b2 = a**2, b**2
        a, b = 2*(a2.shift(1) + a*b), b2 - a2
        # and add
        if c == 1:
            a, b = 2*a.shift(1) + b, -a

    # Here's the d-division polynomial
    f = a.shift(1) + b

    # Find a point on the right twist with the right order
    x = False
    while not x:
        x = K.random_element()
        if (x**2 - 1).is_square() == qres:
            if any(pellmul(x, C // p) == 1 for p in primes):
                x = False
        else:
            x = False

    return f - x

# Montgomery ladder for Pell conics
def pellmul(x, n):
    # The zero point and ours
    A, B = 1, x
    for c in reversed(n.digits(2)):
        if c == 0:
            A, B = 2*A**2 - 1, 2*A*B - x
        else:
            A, B = 2*A*B - x, 2*B**2 - 1
    return A
\end{verbatim}


Another parameterization of $T_2$ is due to Rubin and Silverberg
\cite{rubin-silverberg+crypto03}. Let $d$ be a quadratic non-residue
and define the map $\psi:\mathbb{A}^1(\F_q) \to T_2(\F_q)$ as
    
\begin{equation}
  \phi(a) = \frac{a+\sqrt{d}}{a-\sqrt{d}} = \frac{a^2+d}{a^2-d} + \frac{2a}{a^2-d}\sqrt{d}.
\end{equation}
    
Conversely, suppose $\beta = \beta_1+\beta_2\sqrt{d}\in T_2$, with
$\beta\ne\pm 1$ (so $\beta_2\ne 0$). Then

\begin{equation}
  \beta = \frac{1+\beta}{1+\beta_1+\beta_2\sqrt{d}}{1+\beta_1-\beta_2\sqrt{d}}
  =\psi\left(\frac{1+\beta_1}{\beta_2}\right).
\end{equation}
    
Thus we have a birational mapping between $T_2 - \{\pm 1\}$ and
$\mathbb{A}^1 - \{0\}$. This map can be extended to an isomorphism by
sending $1$ to $\infty$ and $-1$ to $0$. A simple calculation shows
that if $a,b\in\F_q$ and $a\ne -b$, then
    
\begin{equation}
  \psi(a) = \psi\left(\frac{ab+d}{a+b}\right).
\end{equation}

\textbf{Note:} I haven't checked it, but I am persuaded that this
group law on the projective line comes from the classical way of
mapping the circle onto it. Adapting the previous algorithm to this
case should be straightforward.

$T_1$ and $T_2$ are the only $1$-dimensional tori. The next step would
be to handle $T_3$, $T_6$ and $T_4$. Rubin and Silverberg, to
construct their cryptosystem CEILIDH \cite{rubin-silverberg+crypto03},
give some rational parameterizations of $T_6$ using a birational
mapping to $\mathbb{A}^2$. These mappings are not group morphisms and
still require to do all the computations in $\F_{q^6}^\ast$, so they
are of no interest to us.

Parameterizing $T_n$ in general is not hard: write the restriction of
scalar from $\F_{q^n}$ to $\F_q$ of $\mathbb{G}_m$, throw in the
equation saying that all elements must have norm $1$, and you're
done. The problem with this representation is that it is suboptimal
(using $n$ coordinates and $n$ equations instead of $m$ and
$m-\phi(n)$), and that it is hardly a convenient one for computations.

The same cases could be handled with supersingular elliptic curves,
thus avoiding point counting. Notice indeed that $\Phi_3(q)=q^2+q+1$,
$\Phi_6(q)=q^2-q+1$, $\Phi_4(q)=q^2+1$, and that for each of these
cases there is a supersingular curve defined over $\F_{q^2}$ (with
some exceptions depending on the order modulo $3$ or $4$ of the prime
$p$ dividing $q$). Write down the equation of this curve, Weil descend
it over $\F_q$ and you're done. This is better in terms of number of
variables ($4$ of them) and equations ($2$ of them), but it is still
rather messy as computations are concerned.

I will argue later why I think there is no hope of successfully using
higher dimensional groups this way, but for the moment I'll just keep
focusing on $T_1$ and $T_2$.

\subsection{Another parameterization of $T_2$}
    
Let's go back to Lenstra's and De Smit's construction
\cite{lenstra+desmit08-stdmodels}. The basic idea is very simple: if
$\ell^\alpha$ divides exactly $q-1$, take an $\ell^\alpha$-th root of
unity $\eta_0$ and construct the tower of extensions
$\eta_1^\ell=\eta_0$, $\eta_2^\ell=\eta_1$ and so on.

Compared with just taking a non $\ell$-adic residue and construct the
tower of its $\ell^\beta$-th roots, this construction has the
advantage of being deterministic, and can be even made canonical by
fixing an arbitrary (e.g., lexicographic) ordering on the
$\ell^\alpha$ roots of unity. But, since we are not really interested
in this level of canonicity, we'll just forget it in this case.

When $\ell$ does not divide $q-1$, things get more interesting. In
this case the cyclotomic polynomial splits into factors of degree
$\ord_\ell(q)$. Pick one of those (this can be done canonically by an
arbitrary ordering, too) and construct its residue class field over
$\F_q$. Call this residue field $L$. Pick any of the roots of this
polynomial, say $\eta_0$, and construct the tower of extensions
$\eta_1^\ell=\eta_0$, $\eta_2^\ell=\eta_1$, etc. This tower is not the
one we were looking for, but setting $\theta_1 =
\Tr_{\F_{q^p}}\eta_1$, $\theta_2 = \Tr_{\F_{q^{p^2}}}(\eta_2)$, etc.,
the $\theta_i$ generate the $\ell$-adic closure of $\F_q$. So one just
needs to find the minimal polynomials of those $\theta$'s and is done.

Shoup had already given an almost identical construction in
\cite{shoup94}. Once the splitting field of $\Phi_\ell$ is
constructed, instead of constructing the tower over one of its roots,
he takes a random $\ell$-adic residue, thus loosing determinism. If he
does so, it is mainly to present both the $\ell|q-1$ and
$\ell\not|q-1$ cases under the same hood, so we may as well be content
with taking a root of $\Phi_\ell$ in Shoup's algorithm.

Lenstra and De Smit view both of the above constructions as the
reduction modulo $p$ of the cyclotomic $\Z_\ell$-extension of $\Q$
\cite{lang1990cyclotomic,washington1997introduction}. This is the
unique extension of $\Q$ with Galois group isomorphic to the
$\ell$-adic integers, and is constructed as the fixed field of
$\Q(\zeta_\ell, \zeta_{\ell^2}, \ldots)$ under the action of the
torsion part of its Galois group. The elements $\eta$ are called
Gauss' periods, and are constructed as traces of the elements of the
generators of $\Q(\zeta_\ell, \zeta_{\ell^2}, \ldots)$ down to some
subfield.

\textbf{Note:} For algorithms computing the minimal polynomials of
Gauss' periods, see
\cite{gupta+zagier93,gurak06powers,gurak06,thaine01,thaine00}.
However none of these papers gives an algorithm good for our
purposes. Shoup's algorithm \cite{shoup94} remains the best way to
compute these irreducible polynomials modulo $p$. Just for a degree
$\ell$ polynomial, its complexity is

\begin{itemize}
\item $\tildO(\ell^2\log q)$ when $q$ has high order modulo $\ell$,
\item $\tildO(\ell^{\frac{\omega+1}{2}} + \ell\log q)$ when the order
  is small (it obviously uses modular composition in this case),
\item $\tildO(\ell + \log q)$ when $\ell$ divides $q-1$.
\end{itemize}
Let's now focus on the case where $\ell$ divides $q+1$, so that
$\Phi_\ell$ splits into quadratic factors in $\F_q$. Let $Q_0$ be any
of these factors and let $\eta_0$ be one of its roots. Then
$N(\eta_0)=\eta_0^{q+1}=1$ and

\begin{equation}
  Q_0(X) = X^2 - X\Tr_{\F_q}\eta_0 + 1.
\end{equation}

Construct the tower of extensions $\eta_1^\ell=\eta_0$,
$\eta_2^\ell=\eta_1$, etc.  Set $\theta_0 = \Tr\eta_0 \in \F_q$.  Like
before, take the traces of the $\eta_i$'s down to a well chosen
subfield. In this case:

\begin{equation}
  \theta_i = \Tr_{\F_{q^{\ell^i}}}\eta_i = \eta_i + \eta_i^{q^{\ell^i}} = \eta_i + \eta_i^{-1}.
\end{equation}
    
Then $\F_{q^{\ell^i}} = \F_q(\theta_i)$. By the equation above, the
minimal polynomial of $\eta_i$ over $\F_q(\theta_i)$ is $X^2
-\theta_iX+1$. By composition, it follows that the minimal polynomial
of $\eta_i$ over $\F_q(\theta_{i-j})$ is
$X^{2\ell^j}-\theta_{i-j}X^{\ell^j}+1$. We deduce then that $Q_{i,j}$,
where

\begin{equation}
  Q_{i,j}(Y)^2 = \mathrm{Res}_X(X^{2\ell^j}-\theta_{i-j}X^{\ell^j}+1, X^2-YX+1),
\end{equation}

is the minimal polynomial of $\theta_i$ over
$\F_q(\theta_{i-j})$. Indeed, by the well known properties of the
resultant,

\begin{multline}
  Q_{i,j}(Y)^2 = \prod_{\sigma\in\Gal_{\F_q(\eta_i)/\F_q(\theta_{i-j})}}
  (\eta_i^2 - Y\eta_i + 1)^\sigma =\\
  % 
  N_{\F_q(\theta_{i-j})}(\eta_i) \prod_{\sigma\in\Gal_{\F_q(\eta_i)/\F_q(\theta_{i-j})}} 
  (\eta_i + \eta_i^{-1} - Y)^\sigma =\\
  % 
  \left(\prod_{\sigma\in\Gal_{\F_q(\eta_i)/\F_q(\theta_{i-j})}/\langle\tau\rangle}
    (\theta_i - Y)^\sigma\right)^2 =\\
  % 
  \left(\prod_{\sigma\in\Gal_{\F_q(\theta_i)/\F_q(\theta_{i-j})}}
    (\theta_i - Y)^\sigma\right)^2,
\end{multline}

where the third equality is true because
$N_{\F_q(\theta_{i-j})}(\eta_i)=1$, as seen from the minimal
polynomial of $\eta_i$, and where $\tau$ is the element of
$\Gal_{\F_q(\eta_i)/\F_q(\theta_{i-j})}$ such that
$\tau:\eta_i\mapsto\eta_i^{-1}$.

Because of its very special form, this resultant can be computed
efficiently, probably in $O(\ell^j)$. The polynomials $Q_{i,1}$
are of special interest to us, because they define each level of the
tower, and are fundamental for the going up and down operations we
will discuss next. As evident from the formula, they depend on
$\theta_{i-1}$ and they all have the same form. To compute their
general shape, we just need to compute
\begin{equation}
  Q_\ell(Y, z) = \sqrt{\mathrm{Res}_X(X^{2\ell}-zX^\ell+1, X^2-YX+1)}.
\end{equation}

Let $\eta$ and $1/\eta$ be the two roots of $X^2-YX+1$, by direct
calculation we find
\begin{equation}
  \label{eq:1}
  Q_\ell = \eta^\ell + \frac{1}{\eta^\ell} - z.
\end{equation}
The factor $p_\ell=\eta^\ell + 1/\eta^\ell$ is the $\ell$-th power sum
of $X^2-YX+1$, thus

\begin{align}
  p_0 &= 2,\\
  p_1 &= Y,\\
  \label{eq:simple}
  p_{\ell+1} &= Yp_{\ell} - p_{\ell-1},\\
  \label{eq:double}
  p_{2\ell} &= p_{\ell}^2 - 2,\\
  \label{eq:add}
  p_{2\ell+1} &= p_{\ell+1}p_\ell - Y.
\end{align}

From equation \eqref{eq:simple} we can derive a formula for the
coefficients of $Q_\ell$. Indeed, writing $p_\ell = \sum_i
c_{\ell,i}Y^{\ell-i}$, we obtain the relation
\begin{equation}
  c_{\ell,i} = c_{\ell-1,i} - c_{\ell-2,i-2}.
\end{equation}
We arrange these coefficients like in Pascal's triangle:
\begin{equation}
  \begin{tabular}{*{9}{c@{}}}
    &&&&2\\
    &&&1&&0\\
    &&1&&0&&-2\\
    &1&&0&&-3&&0\\
    1&&0&&-4&&0&&2
  \end{tabular}
\end{equation}
It is then evident that every second column is identically zero, and
that signs alternate in odd columns. We remove the zero columns and
take absolute values, i.e., we set $b_{n,k}=\lvert c_{n+k,2k}\rvert$, so
that the new coefficients satisfy the relation
\begin{equation}
  b_{n,k} = b_{n-1,k} + b_{n-1,k-1},
\end{equation}
which is the same as Pascal's relation. Indeed, we obtain the
$(1,2)$-Pascal triangle, also called Lucas' triangle~\cite{benjamin10}:
\begin{equation}
  \begin{tabular}{*{9}{c@{}}}
    &&&&2\\
    &&&1&&2\\
    &&1&&3&&2\\
    &1&&4&&5&&2\\
    1&&5&&9&&7&&2
  \end{tabular}
\end{equation}

It is well known that the coefficients of Lucas' triangle are related
to those of Pascal's by
\begin{equation}
  \label{eq:lucas-tri}
  b_{n,k} = \binom{n}{k} + \binom{n-1}{k-1} = \frac{n+k}{n}\binom{n}{k}.
\end{equation}
One remarkable property of the Lucas triangle is that the
anti-diagonals sum up to the Lucas numbers:
\begin{equation}
  \sum_{k\ge0} b_{n-k,k} = \sum_{k\ge0} \lvert c_{n,2k}\rvert = L_n.
\end{equation}
An immediate consequence of this property is the remarkable equality
\begin{equation}
  p_\ell(i) = \sum_{k\ge0} c_{\ell,k}i^{\ell-k} = i^\ell L_\ell.
\end{equation}

According to OEIS and Plouffe's thesis, other notable properties are
that the columns contain the coefficients of some Chebyshev
polynomials\footnote{OEIS A5581, A5582, A5583, A5584. Which kind of
  Chebyshev polynomials these are, I really don't know.}:
\begin{equation}
  \sum_{n\ge0}b_{n+k,k}z^n = \sum_{n\ge0}\lvert c_{n+2k,2k}\rvert z^n = \frac{2-z}{(z-1)^{k+1}},
\end{equation}
and that the diagonals contain the $n$-dimensional pyramidal
numbers\footnote{OEIS A330, A2415, A5585.}:
\begin{equation}
  \sum_{k\ge0} b_{n+k,k}z^k \sum_{k\ge0} \lvert c_{n+2k,2k} \rvert z^k =
  \frac{1+z}{(z-1)^{n+1}}.
\end{equation}

By using Eq.~\eqref{eq:lucas-tri} and the sign alternation property,
we obtain 
\begin{gather}
  c_{\ell,2k} = (-1)^kb_{\ell-k,k} = (-1)^k\frac{\ell}{\ell-k}\binom{\ell-k}{k},\\
  \frac{c_{\ell,2k+2}}{c_{\ell,2k}} = 
  -\frac{(\ell-2k)(\ell-2k-1)}{(\ell-k-1)(k+1)}.
\end{gather}
The last equation allows us to compute all the coefficients of
$Q_\ell$ using $O(\ell)$ operations in the base field. For example, we
have
\begin{align}
  Q_5 &= Y^5 - 5Y^3 + 5Y - z,\\
  Q_7 &= Y^7 - 7Y^5 + 14Y^3 - 7Y - z,\\
  Q_{11} &= Y^{11} - 11Y^9 + 44Y^7 - 77Y^5 + 55Y^3 - 11Y - z.
\end{align}

These facts are not new: a special case was already settled by Gauss,
and the general case is in \cite[Proposition~1]{gurak06}.

To go back to the geometric point of view, observe that
equations~\eqref{eq:double} and~\eqref{eq:add} are almost identical to
the equations defining the group law on the torus. This is no
coincidence, as we will show that, by seeing $\F_q(\eta_i)^\ast$ as a
degree $2$ extension over $\F_q(\theta_i)$, the $\ell$-th power map is
just the multiplication-by-$\ell$ map on $T_1\oplus T_2$, and it is
straightforward to obtain a parameterization of $T_2/\F_q(\eta_i)$ as
a Pell conic.

Let $\sigma$ be the non trivial element of the Galois group of
$\F_q(\eta_i)/\F_q(\theta_i)$, so that
$\sigma(\eta_i)=\eta_i^{-1}$. Set $d = \eta_i - \sigma(\eta_i)$. Then
$\sigma(d)=-d$ and $d^2 = \theta_i^2-4$, so that

\begin{equation}
  \F_q(\eta_i) = \F_q(d) = \F_q\left(\sqrt{\theta_i^2-4}\right).
\end{equation}

Every element $\alpha\in\F_q(\eta_i)$ can be uniquely written as
$x + yd$ with $x,y\in\F_q(\theta_i)$, where

\begin{align}
  x &= \frac{\alpha + \sigma(\alpha)}{2},\\
  y &= \frac{\alpha - \sigma(\alpha)}{2}\frac{d}{\theta_i^2-4}.
\end{align}
    
An element of $\F_q(\eta_i)$ belongs to $T_2$ if and only if its norm
is $1$, i.e. if and only if

\begin{equation}
  N(x + yd) = (x + yd)(x - yd) = x^2 - (\theta_i^2-4)y^2 = 1.
\end{equation}

Thus we have defined an isomorphism between
$T_2\subset\F_q(\eta_i)^\ast$ and a Pell conic defined over
$\F_q(\theta_i)$. The neutral element of $T_2$ is sent over $(1,0)$,
and we check that the morphism is a group morphism: by the definition
of the previous section

\begin{equation}
  (x,y) \oplus (x',y') = (xx' + (\theta_i^2-4)yy', xy' + x'y),
\end{equation}

and, similarly,

\begin{equation}
  (x + yd)(x' + y'd) = xx' + (\theta_i^2 - 4)yy' + (xy' + x'y)d,
\end{equation}

as needed. Let now $[\ell]$ denote multiplication by $\ell$ on the
conic and suppose that

\begin{equation}
  [\ell](x,y) = \left(f_\ell(x), g_\ell(x, y)\right).
\end{equation}

Observe that

\begin{align}
  \eta_i &= \frac{1}{2}\theta_i + \frac{1}{2}d,\\
  \eta_{i-1} &= \eta_i^\ell = f_\ell\left(\frac{1}{2}\theta_i\right) + g_\ell\left(\frac{1}{2}\theta_i, \frac{1}{2}\right)d,
\end{align}
    
Then, since $\sigma(\eta_{i-1}) = \eta_{i-1}^{-1}$, we have

\begin{equation}
  f_\ell\left(\frac{1}{2}\theta_i\right) = \frac{1}{2}\theta_{i-1}.
\end{equation}

By the previous section, we know that $f_\ell$ has degree $\ell$, thus

\begin{equation}
  2f_\ell\left(\frac{Y}{2}\right) - \theta_{i-1}
\end{equation}

is the minimal polynomial of $\theta_i$.

\subsection{Up and down in the one dimensional case}

Summarizing, when $\ell$ divides $q^2-1$, using Couveignes' and
Lercier's idea on either $T_1$ or $T_2$ we obtain a series of
irreducible polynomials of the form

\begin{equation}
  f(X_i) - X_{i-1},
\end{equation}

defining the $\ell$-adic closure of $\F_q$. We have also seen that
Lenstra's and De Smit's construction yields exactly the same families.

When $\ell$ does not divide $q^2-1$, we can find an elliptic curve
such that its cardinality is a multiple of $\ell$ and use the original
Couveignes' and Lercier's algorithm. This can be done efficiently only
if $\ell\ll q^{1/4}$, and in general only if $\ell<q+2\sqrt{q}+1$
because of Hasse's bound. In this case, the family of irreducible
polynomials obtained is of the form

\begin{equation}
  f_i(X_i) - X_{i-1}g_i(X_i).
\end{equation}

We call these cases the \emph{one dimensional} cases because indeed we
are using one dimensional varieties to construct these families. An
elementary combinatorial argument shows that the one dimensional case
can only deal with $\ell\in O(q)$, because a one dimensional variety
over $\F_q$ cannot have more points.

It turns out that the two families above (and intuitively, any family
issued by a one dimensional object) are perfectly suited for going up
and down in the tower. A metatheorem we know well says that once a
quasi-linear algorithm is known for going down (resp. up), going up
(resp. down) follows immediately with the same complexity using the
classical works of Rouiller, Shoup, and others (see Chapter 5 of my
thesis for a review \cite{df+thesis}). However, we will give explicit
algorithms for both operations.

\subsubsection{Using algebraic tori}

Let's consider the algebraic tori case first. We have
$\F_{q^{\ell^i}}$ represented as an univariate extension of $\F_q$,
and we want to take one of its elements $P(X_i)$ down to the basis
bivariate $\F_{q^{\ell^{i-1}}}[X_i]$. All it takes is to compute

\begin{equation}
  P(X_i) \mod f(X_i) - X_{i-1}, Q(X_{i-1})
\end{equation}

and we are done. Using Algorithm 9.14 of von zur Gathen and Gerhard
\cite{vzGG}, we rewrite $P(X_i)=\sum_j a_j(X_i)f(X_i)^j$ with $\deg
a_j<\ell$ in quasi linear time, then the reduction is $\sum_j
a_j(X_i)X_{i-1}^j$. Algorithm 9.14 is as simple as the following Sage
code:

\begin{verbatim}
def decompose(P, f):
    if P.degree() < f.degree() * 2:
        return P.quo_rem(f)
    else:
        return sum((g.quo_rem(f) for g in decompose(P, f^2)), ())
\end{verbatim}

To go the opposite direction, we could use the machinery of trace
formulas, but it is very easy to do it directly. Our input is a
bivariate polynomial $P(X_{i-1},X_i)$ of degree $<\ell$ in $X_i$ and
$<\ell^{i-1}$ in $X_{i-1}$, and we want to reduce it modulo the
relation $X_{i-1}=f(X_i)$, thus we need to compute
\begin{equation}
  P(f(X_i), X_i).  
\end{equation}
Horner rule is not optimal in this case, but a binary divide and
conquer approach like in the previous case has the right complexity:
$O(\ell)$ at the leaves (there is $\ell^{i-1}$ of them) and
$O(\Mult(\ell^i))$ at each level. Here's one way of coding the
algorithm in Sage (\verb+P+ is a list of lenght $\le\ell^{i-1}$ of
polynomials of degree $<\ell$ in $X_i$).

\begin{verbatim}
def compose(P, f):
    if len(P) == 0:
        return f.parent()(0)
    elif len(P) == 1:
        return f.parent()(P[0])
    else:
        g = f^2
        return compose(P[::2], g) + f*compose(P[1::2], g)
\end{verbatim}


\subsubsection{Using elliptic curves}
    
If we are in the elliptic curve case, now, we can rewrite the
equation above as

\begin{equation}
  \label{eq:elliptic-relation}
  \frac{f_i(X_i)}{g_i(X_i)} = X_{i-1}.
\end{equation}

I find it easier to deal with going up in this case. Like before, we
have a polynomial $P(X_{i-1},X_i)$ and we want to compute
\begin{equation}
  P\left(\frac{f(X_i)}{g(X_i)}, X_i\right).
\end{equation}
From now on, we view $P$ as a polynomial in $X_{i-1}$ with coefficients in
$\F_q[X_i]$. Let
\begin{equation}
  P = \sum_{j=0}^n p_j(X_i)X_{i-1}^j,
\end{equation}
and let
\begin{equation}
  \bar{P} = g^n P\left(\frac{f}{g}\right) =
  \sum_{j=0}^n p_jf^jg^{n-j}.
\end{equation}
The univariate polynomial $\bar{P}$ can be computed recursively as
follows. We suppose $n$ is even to simplify matters, let
$P=P_0+X_{i-1}^{n/2+1}P_1$ with $\deg P_0\le n/2$, and let $\bar{P}_0$
and $\bar{P}_1$ be defined as above, then
\begin{equation}
  \bar{P} = f^{n/2}\bar{P}_0 + g^{n/2+1}\bar{P}_1.
\end{equation}
Thus, to compute $\bar{P}$ we need $O(\Mult(\ell^i)\log_2\ell^i)$
operations as in the case of the algebraic torus. After we compute
$\bar{P}$, we can compute $g^n$ using $O(\Mult(\ell^i))$ operations,
interpret it as an element of $\F_{q^{\ell^i}}$ and invert it, then
obtain the final result by multiplying $\bar{P}$ by $g^{-n}$.

Going down can be done with about the same complexity using trace
formulas. It took me a while to figure out how to do it directly. In
this case, our input is a polynomial $P(X_i)$ of degree $<\ell^i$,
and we want to reduce it modulo Eq.~\eqref{eq:elliptic-relation}. By
analogy with the previous algorithm, it seems natural to compute
$\bar{P}=g^nP$, where $n=\ell^{i-1}-1$, and then try to recursively
decompose $\bar{P}$ as
\begin{equation}
  \bar{P} = f^{m}\bar{P}_0 + g^{m}\bar{P}_1.
\end{equation}

We start by solving a simpler problem. Let $a,b$ be two polynomials
with $\deg a\ge \deg b$ ($b$ can eventually be $1$), and let $P$ be
another polynomial with degree less than $2\deg a$. Our goal is to
find an expression
\begin{equation}
  P = aP_0 + bP_1  
\end{equation}
with $\deg P_0$ and $\deg P_1$ less than $\deg a$. Observe that if
$\deg P<\deg a+\deg b$, then the result is easily obtained by an
inverse Chinese remainder. Indeed, let
\begin{equation}
  ua + vb = 1
\end{equation}
with $\deg u<\deg b$ and $\deg v<\deg a$, then
\begin{equation}
  P = (Pu\bmod b)a + (Pv\bmod a)b.
\end{equation}
When $\deg P \ge \deg a+\deg b$, handling the extra factors is as
easy. By Euclidean division, write
\begin{equation}
  P = Qa + R,
\end{equation}
then
\begin{equation}
  P = \bigl(Q + (Ru\bmod b)\bigr)a + (Rv\bmod a)b.
\end{equation}
Although this decomposition is not unique, the degree constraints are
satisfied. Observe that, when $b=1$, this amounts to decompose $P$ as
$Qa+R$. I must thank Antoine Joux for helping me out on this.

Now back to the original problem. We want to write, if possible,
$\bar{P}$ as
\begin{equation}
  \bar{P} = \sum_{j=0}^n p_jf^jg^{n-j},
\end{equation}
with $\deg p_j<\deg f$, where $n$ is a parameter fixed in advance. The
idea is to compute powers of $f$ and $g$ whose degrees sum up to $n$,
and to use the algorithm above to iteratively decompose $\bar{P}$. If
$n$ is even, we let $2m=n$; if $n$ is odd, we let $2m=n+1$. In either
case, we compute $f^m$ and $g^m$ and decompose $\bar{P}$ as
\begin{equation}
  \bar{P} = f^mP_0 + g^mP_1.
\end{equation}
Now we apply recursively the algorithm to decompose $P_0$ and $P_1$
as
\begin{equation}
  P_0 = \sum_{j=0}^{n-m} p_{0,j}f^jg^{n-m-j},\qquad
  P_1 = \sum_{j=0}^{n-m} p_{1,j}f^jg^{n-m-j}.
\end{equation}
With a bit of tedious work, we verify that the constraint on the
degrees of the $p_j$ can be met if $\deg P<(n+1)\deg f$. This is
exactly what we want, indeed, fixing $n=\ell^{i-1}-1$ as target
degree, we verify $\deg P < \ell^i = (n+1)\ell$. Observe that when
$g=1$, this algorithm reduces to a variant of Algorithm 9.14.

\subsection{Implementation}
I have implemented the algorithms above in the toric and the elliptic
case (computing the irreducible polynomials, building the tower, going
up and down). I have written a very simple Sage implementation, but
this is enough to consistently outperform Magma (I shall need a recent
version of Magma, though). The toric case is just a few dozen of lines
of code, and immensely beats Magma for any parameter choice. 

The elliptic case is trickier. While going up and down is almost as
efficient as the toric case, there is a serious precomputation to
perform when initializing the tower. This is made of two major steps:

\begin{enumerate}
\item Find a curve with the right cardinality,
\item Compute the whole $\ell$-isogeny volcano.
\end{enumerate}

The first step is done with point counting. Arguably, one could count
points modulo $\ell$, thus, in some cases, the cost could be (a large)
polynomial in $\ell$. In general, however, this is done by a full
point counting.

The necessity for the second step occurred to me as a
surprise. Ideally, one would want to compute isogenies one at a time,
by adding one for each level. However, we need to compute the
isogenies with rational kernel. The easiest way to do this is to
repeatedly apply Velu's formulas, but this gives a chain of isogenies
\emph{starting} from $E$, while we need a chain \emph{ending} in
$E$. So we are bound to compute the whole isogeny cycle, and stop when
we hit the starting curve again. The length of the cycle is equal to
the order of a certain ideal class, this is at most the class number
of the endomorphism algebra of the curve. In general, the class number
is $\sim\sqrt{q}$, thus we have another polynomial factor in $q$
intervening in the complexity.

An alternative approach to compute the isogenies would be to factor
the $\ell$-th modular polynomials in order to follow the cycle
backwards. This has the advantage of not depending on $q$, but it
costs $O(\ell^3)$ at least. It looks surprising that this isogeny
cycle is very easy to compute in one direction, but extremely hard in
the dual direction; however, I have no other idea on how to compute
these isogenies backwards.


\subsection{Higher dimensions}
    
The last question is: what can we do when $\ell$ is much bigger than
$q$? I think there is no hope of making any of the previous techniques
work in this case, and I will try to argue on this a bit more
seriously sometimes later.


\section{Bibliography \textbf{:export:}}

\printbibliography

\end{document}


% Local Variables:
% mode:flyspell
% ispell-local-dictionary:"american"
% mode:TeX-PDF
% mode:reftex
% End:
%
% LocalWords:  Isogeny abelian isogenies hyperelliptic supersingular Frobenius
% LocalWords:  isogenous
