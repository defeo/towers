\documentclass{sig-alternate}

\usepackage[utf8]{inputenc}
\usepackage[T1]{fontenc}
\usepackage{graphicx}
\usepackage{amssymb}
\usepackage{amsmath}
\usepackage{mysymbols}
\usepackage{hyperref}

\usepackage{xcolor}
\newcommand{\todo}[1]{\textcolor{red}{(#1)}}

\begin{document}

\title{Cool title to catch Lenstra's eye goes here}
\numberofauthors{3}
\author{
  \alignauthor Luca De Feo
  \alignauthor Javad Doliskani
  \alignauthor Éric Schost
}

\maketitle
\begin{abstract}
  We build towers.
\end{abstract}
\category{F.2.1}{Theory of computation}{Analysis of algorithms and problem complexity}[Computations in finite fields]
\category{G.4}{Mathematics of computing}{Mathematical software}
\terms{Algorithms,Theory}
\keywords{Finite fields, irreducible polynomials, extension towers, algebraic tori, elliptic curves.}

%%%

\section{Introduction}
\label{sec:intro}

Building arbitrary finite extensions of finite fields is a fundamental
task in any computer algebra system. Disregarding generic solutions
using multivariate algebra and Grobner bases, most systems are limited
to construct extensions of prime fields. There is one notable
exception: Magma~\cite{MAGMA} has implemented for a long time a
powerful system of ``compatibly embedded finite
fields''~\cite{bosma+cannon+steel97}, capable of building extensions
of any finite field and taking track of the embeddings between the
fields.

The system implemented in Magma, as described in the original paper,
uses linear algebra to describe the embeddings of finite fields. From
a complexity point of view this is far from optimal, indeed one may
hope to compute and apply the morphisms in (quasi)-linear time (in the
degree of the extension). Even worse, the quadratic memory
requirements make the system unsuitable for embeddings of large degree
extensions. Although the Magma core has evolved since the publication
of the paper, our experiments in Section~\ref{sec:bench} show that
embeddings of large extension fields are still out of reach in Magma.

In this paper we propose an approach based on polynomial arithmetic
rather than linear algebra, yielding much better performances. Our
solution only applies to some special cases, but we hope that it will
pave the way towards a complete solution. The technique we present is
similar to the one in~\cite{df+schost12} in that we construct families
of irreducible polynomials with special properties, then give
algorithms (called \emph{lift} and \emph{push}) that exploit the
special form of those polynomials to apply the embeddings. The
families of polynomials we use come from work of Lenstra and De Smit
on the algebraic closure of finite
fields~\cite{lenstra+desmit08-stdmodels}, and from work of Couveignes
and Lercier on constructing irreducible polynomials in quasi-linear
time~\cite{couveignes+lercier11}.

The focus of this paper is on the representation of extensions of
\emph{large} degree and their embeddings. We do not consider issues
related to special representations of small finite fields, such as
primitive polynomials, Zech logarithms, Conway polynomials, etc. A
user doing computations in a single, fixed, extension field will
clearly benefit more from these optimizations than from our
construction. However, a variety of computations in number theory and
algebraic geometry requires frequently constructing new extension
fields and moving elements from one to the other. Examples of these
algorithms are~\cite{df10} \todo{cite more}. Our construction makes
them feasible for much larger sizes than it was previously possible.

This paper is organized as follows. \todo{How?}

%%%

\section{Finite fields and embeddings}
\label{sec:finite-field-embedd}
Basic theory of finite fields. Standard algorithms, Shoup's algorithm.

%%%

\section{Lenstra's and De Smit's tower}
\label{sec:LDtower}
The $p$-adic extension of $\Q$ and its reduction mod $p$.

%%%

\section{Towers from irreducible fibers}
\label{sec:fibers}
Interpreting the primitive roots tower and the LD-tower as fibers of
$T_1$ and $T_2$ respectively. Using elliptic curves.


%%%

\section{Lifting and pushing}
\label{sec:lift-push}

The two algorithms given in the notes for the rational fraction case
(and how they reduce to the polynomial case).

%%%

\section{The general case}
\label{sec:general}

Composita, general LD-towers, other algebraic tori, higher dimensional
fibers...

%%%

\section{Benchmarks}
\label{sec:bench}

Sage implementation and timings.

%%%

\section{Aknowledgements}
De Feo would like to thank Antoine Joux.

\bibliographystyle{abbrv}
\bibliography{defeo}
\end{document}


% Local Variables:
% mode:flyspell
% ispell-local-dictionary:"american"
% mode:TeX-PDF
% mode:reftex
% End:
%
% LocalWords:  Isogeny abelian isogenies hyperelliptic supersingular Frobenius
% LocalWords:  isogenous embeddings morphisms
